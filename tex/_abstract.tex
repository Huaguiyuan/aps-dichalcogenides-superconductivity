\begin{abstract}
  Strong spin orbit interaction has the potential
  to engender unconventional superconducting states.
  Two dimensional dichalcogenides, MX$_{2}$
  where M is a transition metal such as Mo or W, and X is a S or Se,
  are particularly interesting as the noninteracting electronic states have
  multiple valleys in the energy dispersion and are topologically nontrivial.
  We investigate the possible superconducting states
  of hole doped systems and analyze to what extent the correlated phase
  inherits the topological aspects of the parent crystal.
  We find that local attractive interactions and proximal coupling to
  s-wave superconductors lead to a pairing which is an equal admixture
  of spin singlet and m=0 spin triplet.
  The valley contrasting optical response,
  where oppositely circularly polarized light couple to different valleys,
  is present even in the superconducting state, but with smaller magnitude.
  Thus for a given polarization pair breaking results
  in one quasiparticle in the conduction band in one valley
  and its partner in the valence band of the other.
  The locking of spin to momentum also results in unusual response
  to magnetic field in that no pair breaking occurs
  for fields in the plane of the crystal in the absence of disorder.
  As such the critical magnetic fields are quite large
  compared to conventional superconductors.
\end{abstract}
