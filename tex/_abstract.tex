\begin{abstract}
  Strong spin orbit interaction has the potential
  to engender unconventional superconducting states.
  Two dimensional dichalcogenides,
  \ce{MX2} ($\ce{M} = \ce{Mo}, \ce{W}$
  and $\ce{X} = \ce{S}, \ce{Se}, \ce{Te}$),
  are particularly interesting: the noninteracting electronic states have
  multiple valleys in the energy dispersion and are topologically nontrivial.
  We report on the possible superconducting states
  of hole-doped systems, and analyze to what extent the correlated phase
  inherits the topological aspects of the parent crystal.
  We find that local attractive interactions or proximal coupling to
  $s$-wave superconductors lead to a pairing which is an equal mixture
  of spin singlet and $m = 0$ spin triplet.
  The valley contrasting optical response,
  where oppositely circularly polarized light couples to different valleys,
  is present even in the superconducting state but with smaller magnitude.
  The locking of spin to momentum results in an unusual response
  to a magnetic field.
  In the absence of disorder, pair-breaking does not occur
  for fields in the plane of the crystal.
  Consequently, the critical magnetic fields are quite large
  compared to conventional superconductors.
  These results demonstrate the rich correlated phenomena possible
  due to the interplay of spin-orbit coupling, interaction, and topology.
\end{abstract}
