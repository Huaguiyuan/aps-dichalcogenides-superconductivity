\section{Berry curvature}

The Berry curvature in the non-interacting crystal
for left and right-circularly polarized
($\vc{ϵ}_±$) optical excitations for a given $\vK$
is $± 2 Ω_{+ ↑}^+ \of{k}$, where
\begin{subequations}
  \begin{align}
    Ω_{τ {\s}}^n \of{k}
    & = \vc{\hat{z}} · \vc{Ω}_{τ {\s}}^n \ofK, \\
    & = - n τ
        \left[ \frac{1}{2 k} \pderiv{}{k} \fnTheta{n} \right]
        \sin{\fnTheta{n}}, \\
    & = - n τ
        \frac{2 {\left( a t \right)}^2 \left( Δ - λ τ {\s} \right)}
        {{\left[{\left( 2 a t k \right)}^2
      + {\left( Δ - λ τ {\s} \right)}^2 \right]}^{3/2}}.
  \end{align}
\end{subequations}

The BCS ground state %
\footnote{%
  Note that the full ground state
  also contains the two lower filled bands,
  but those contribute zero net Berry curvature and may be ignored
  in this section and the next.}
is
\begin{subequations}
  \begin{align}
    \Ket{Ω}
    & = ∏_{\vK} \csc{β_{\vK}} b_{\vK ↑} b_{-\vK ↓} \Ket{0}, \\
    & = ∏_{\vK} \left( \cos{β_{\vK}} - \sin{β_{\vK}}
        c_{\vK ↑}^† c_{-\vK ↓}^† \right) \Ket{0}.
  \end{align}
\end{subequations}
This superconducting state is built up
from the quasiparticle eigenstates,
$\Ket{\vK}
= \csc{β_{\vK}} b_{\vK ↑} b_{-\vK ↓} \Ket{0}$,
of the $\vK$-dependent Hamiltonian
$λ_{\vK} \left( b_{\vK ↑}^† b_{\vK ↑}
+ b_{-\vK ↓}^† b_{-\vK ↓} \right)$.
The $z$-component of the Berry curvature of
the correlated state is zero,
\begin{equation}
  \vc{\hat{z}} · i ∇_{\vK} ⨯
  \Braket{\vK | ∇_{\vK} | \vK}
  = Ω_{+ ↑}^- \of{k} + Ω_{- ↓}^- \of{-k} = 0.
\end{equation}
A single optically excited state in the left valley
for a given $\vK$ is
${c_{+ ↑}^+}^† \ofK c_{+ ↑}^- \Ket{\vK}$,
which has a Berry curvature
$+2 \sin^6 {β_{\vK}} Ω_{+ ↑}^+ \of{k}$.
The corresponding excitation in the right valley
has a Berry curvature of the same magnitude but opposite sign.
This contrast, combined with the valley selective optical transition rate,
predicts that the superconducting state
will also exhibit the valley Hall effect.
