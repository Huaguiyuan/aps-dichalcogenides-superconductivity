\section{Berry curvature}

The Berry curvature of left and right circularly polarized
($\vc{ϵ_±}$) optical excitations for a given $\vK$
is $± 2 Ω_{+ ↑}^+ \of{k}$, where
\begin{subequations}
  \begin{align}
    Ω_{τ σ}^n \of{k},
    & = \vc{\hat{z}} · \vc{Ω}_{τ σ}^n \ofK, \\
    & = - n τ
      \left[ \frac{1}{2 k} \pderiv{}{k} \fnTheta{n} \right]
      \sin{\fnTheta{n}}, \\
    & = - n τ
      \frac{2 {(a t)}^2 (Δ - λ τ σ)}
      {{\left[{(2 a t k)}^2 + {(Δ - λ τ σ)}^2 \right]}^{3/2}}.
  \end{align}
\end{subequations}

The BCS ground state %
\footnote{%
  Note that the full ground state
  also contains the lower two filled bands,
  but those contribute zero net Berry curvature and may be ignored
  in this section and the next.}
is
\begin{subequations}
  \begin{align}
    \Ket{Ω}
    & = ∏_{\vK} \csc{β_{\vK}} b_{\vK ↑} b_{-\vK ↓} \Ket{0}, \\
    & = ∏_{\vK} \left( \cos{β_{\vK}} - \sin{β_{\vK}}
        c_{\vK ↑}^† c_{-\vK ↓}^† \right) \Ket{0}.
  \end{align}
\end{subequations}
This BCS ground state may be viewed as built up
from the quasiparticle eigenstates,
$\Ket{\vK}
= \csc{β_{\vK}} b_{\vK ↑} b_{-\vK ↓} \Ket{0}$,
of the $\vK$-dependent Hamiltonian
$- λ_{\vK} \left( b_{\vK ↑} b_{\vK ↑}^†
+ b_{-\vK ↓} b_{-\vK ↓}^† \right)$.

As expected, the $z$-component of the Berry curvature of this state
is zero,
\begin{subequations}
  \begin{align}
    Ω_{\vK}
    & = \vc{\hat{z}} · i ∇_{\vK} ⨯
    \Braket{\vK | ∇_{\vK} | \vK}, \\
    & = Ω_{+ ↑}^- \of{k} + Ω_{- ↓}^- \of{-k} = 0.
  \end{align}
\end{subequations}

Note that a single optically excited state in the left valley
for a given $\vK$ is
% TODO This notation is not consistent.
${c_{+ ↑}^+}^† \ofK c_{\vK ↑} \Ket{\vK}$,
which has corresponding Berry curvature
$2 \sin^6 {β_{\vK}} Ω_{+ ↑}^+ \of{k}$.
Similarly, a single optically excited state in the right valley
has corresponding Berry curvature identically opposite to this.
Thus, combined with the valley selective optical transition rate,
we expect the superconducting state will also exhibit the valley-hall effect.
