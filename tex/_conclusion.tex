\section{Conclusions}

In this letter, we report on the nature of the superconducting state
of hole-doped TMDs.
Remarkably, the correlated state inherits
the valley contrasting phenomena of the non-interacting state.
While the magnitude is smaller, pair-breaking produces quasiparticles
that have the same Berry curvature, and hence the same anomalous velocity.
Thus, one predicts a valley Hall response similar to
the one observed in \ce{MoSe2}.
Further consequences of the phenomena will be explored elsewhere.

Spin-valley locking leads to large critical magnetic fields.
A similar phenomena was recently reported in heavily hole-doped
(beyond the spin-split gap) \ce{NbSe2}
\cite{%
  1510.06289v2,%
  PhysRevLett.113.097001%
}.
In the new regime, where only one band per valley intersects
the chemical potential, no pair-breaking occurs
for in-plane fields unless disorder is present.

While systematic synthesis and characterization of hole-doped systems
is still in its early stages, the fact that other two-dimensional compounds
and their bulk counterparts are known to be superconducting
\cite{%
  PhysRevB.88.054515%
}
provides impetus to explore the novel phenomena described here.
