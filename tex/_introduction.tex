\section{Introduction}

The interplay of spin orbit interaction and electron-electron interaction
is a fertile new area of research where new phases of matter
and novel phenomena have been theoretically conjectured
and experimentally realized.
Single layer transition metal group-VI dichalcogenides (TMDs),
\ce{MX2} ($\ce{M} = \ce{Mo}, \ce{W}$
and $\ce{X} = \ce{S}, \ce{Se}, \ce{Te}$),
are direct band gap semiconductors that have all the necessary ingredients
to explore this interplay
\cite{%
  RadisavljevicB.2011,%
  PhysRevB.84.153402,%
  doi:10.1021/nl2021575,%
  Wang2012,%
  Ye30112012,%
  Bao2013,%
  1.4804936,%
  PhysRevB.88.075409,%
  Xu2014,%
  1508.03068%
}.
While sharing the hexagonal crystal structure of graphene,
they differ in three important aspects:
(1) inversion symmetry is broken, leading to a gap in the spectrum
as opposed to Dirac nodes;
(2) spin is coupled with momenta, resulting in
a large splitting of the valence bands;
and (3) the bands near the chemical potential predominantly have
the transition metal $d$-orbitals character
\cite{%
  0022-3719-5-7-007,%
  PhysRevB.64.235305,%
  PhysRevLett.105.136805,%
  doi:10.1021/nl903868w,%
  PhysRevLett.108.196802,%
  PhysRevB.88.045416,%
  PhysRevB.88.085433%
}.

The nontrivial Berry curvature
associated with the bands near the valleys
is a striking consequence of strong spin-orbit coupling
enabled by inversion symmetry breaking and heavy elements
such as \ce{Mo} and \ce{W}.
The Berry curvature engenders an effective intrinsic angular momentum
associated with the Bloch wave functions.
Remarkably, spin-preserving optical transitions between valence
and conduction bands are possible,
even though the atomic orbitals involved all have a $d$-character.
Furthermore, the valley-dependent sign of
the Berry curvature leads to selective photoexcitation:
right-circular polarization couples to one valley,
and left-circular to the other.
Consequently, this enables a number of valleytronic and spintronic applications
that have attracted a lot of attention over the last few years.
% TODO Need to add references to Di Xiao(PRL), Qian Niu(RMP), Mceuen experiment.

We are primarily interested in exploiting
the band structure and valley-contrasting probe afforded by
the nontrivial topology in order to study and manipulate
correlated phenomena in these systems.
In particular, we focus on hole-doped systems,
where an experimentally accessible window in energy
is characterized by two disconnected pieces of
spin non-degenerate Fermi surfaces.
One can preferentially excite electrons from either Fermi surface.
Since the spins are locked to their valley index,
these excitations have specific $s_z$
(where the $z$-axis is perpendicular to the two-dimensional crystal).
We focus on the possible superconducting states and their properties.

Spin-valley locking and its consequence for superconductivity,
dubbed Ising superconductivity, has been previously studied
for heavily doped $p$-type and $n$-type TMDs
\cite{%
  1510.06289v2,%
  PhysRevLett.113.097001%
},
where Fermi surfaces of each spin are present in each valley.
Here, we focus on the interesting new regime where only one Fermi surface
exists in each valley.
The two valleys in the energy landscape generically allow
two classes of superconducting phases:
intervalley pairing with zero center of mass momentum,
and intravalley pairing with finite Cooper pair center of mass.
The latter case preserves time-reversal symmetry
because there are as many pairs with $+\vc{K}$ as $-\vc{K}$,
where $± \vc{K}$ is the location of the valleys.
Since center-of-symmetry is broken and spin degeneracy is lost,
one expects classifications of superconducting states by parity,
i.e., singlet vs.\ triplet, is no longer possible.
In this paper, we study both extrinsic and intrinsic superconductivity
by projecting the interactions and pairing potential to
the topmost valence band.
We identify the possible phases, and analyze the nature
of the optoelectronic coupling and the response to magnetic fields.

Our main conclusions are as follows.
(1) For both proximity to an $s$-wave superconductor,
and due to local attractive density-density interactions,
the leading instability is do to an intervalley paired state,
where the Cooper pair is an equal mixture of singlet and $m = 0$ triplet.
(2) While the valley selectivity of the optical transition is suppressed,
it remains finite.
Thus, circularly polarized light results in pair-breaking:
a Cooper pair splits into one quasiparticle
in the conduction band of one valley,
while its partner is in the valence band of the other.
Consequently, we predict an anomalous Hall effect of quasiparticles,
but unlike the parent states, both excitations are particle-like.
(3) An in-plane magnetic field tilts the spin,
modifying the internal structure of the Cooper pair,
However, no pair-breaking is induced in the absence of scalar impurities.
The suppression of the effective interaction leads
to a parametric suppression of the transition temperature.
In the presence of scalar impurities, pair-breaking is enabled,
but the associated critical magnetic field is large.
