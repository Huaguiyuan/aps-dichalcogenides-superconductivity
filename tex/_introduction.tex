\section{Introduction}

The interplay of spin-orbit interaction and electron-electron interaction
is a fertile area of research where new phases of matter
and novel phenomena have been theoretically conjectured
and experimentally realized
\cite{%
  PhysRevLett.61.2015,%
  PhysRevLett.95.226801,%
  PhysRevLett.96.106802,%
  Konig02112007,%
  RevModPhys.82.3045,%
  RevModPhys.83.1057,%
  doi:10.1146/annurev-conmatphys-020911-125138% chktex 8
}.
Single-layer transition metal group-VI dichalcogenides (TMDs),
\ce{MX2} ($\ce{M} = \ce{Mo}, \ce{W}$
and $\ce{X} = \ce{S}, \ce{Se}, \ce{Te}$),
are direct band gap semiconductors that have all the necessary ingredients
to explore this phenomena
\cite{%
  RadisavljevicB.2011,%
  PhysRevB.84.153402,%
  doi:10.1021/nl2021575,%
  Wang2012,%
  Ye30112012,%
  Bao2013,%
  1.4804936,%
  PhysRevB.88.075409,%
  Xu2014,%
  1508.03068%
}.
While sharing the hexagonal crystal structure of graphene,
they differ in three important aspects:
(1) inversion symmetry is broken, resulting in a gap
as opposed to Dirac nodes;
(2) spin is coupled to momenta, yielding
a large splitting of the valence bands;
and (3) the bands near the chemical potential predominantly have
the transition metal $d$-orbital character
\cite{%
  0022-3719-5-7-007,% chktex 8
  PhysRevB.64.235305,%
  PhysRevLett.105.136805,%
  doi:10.1021/nl903868w,%
  PhysRevB.88.045416,%
  PhysRevB.88.085433%
}.

The nontrivial Berry curvature
associated with the bands near the valleys
is a consequence of strong spin-orbit coupling
enabled by inversion symmetry breaking and heavy elements
such as \ce{Mo} and \ce{W}.
The Berry curvature engenders an effective intrinsic angular momentum
associated with the Bloch wave functions.
Remarkably, spin-preserving optical transitions between valence
and conduction bands are possible,
even though the atomic orbitals involved all have a $d$-character.
Furthermore, the valley-dependent sign of
the Berry curvature leads to selective photoexcitation:
right circular polarization couples to one valley,
and left circular polarization to the other.
Consequently, this enables a number of valleytronic and spintronic applications
that have attracted a lot of attention over the last few years
\cite{%
  RevModPhys.82.1959,%
  PhysRevLett.108.196802,%
  Mak27062014%
}.

We are primarily interested in exploiting
the band structure and valley-contrasting probe afforded by
the nontrivial topology in order to study and manipulate
correlated phenomena in these systems.
In particular, we focus on hole-doped systems,
where an experimentally accessible window in energy
is characterized by two disconnected pieces of
spin non-degenerate Fermi surfaces.
One can preferentially excite electrons from either Fermi surface.
Since the spins are locked to their valley index,
these excitations have specific $s_z$
(where the $z$-axis is perpendicular to the two-dimensional crystal).
We focus on the possible superconducting states and their properties.

Spin-valley locking and its consequence for superconductivity,
dubbed Ising superconductivity, has been previously studied
for heavily doped $p$-type and $n$-type TMDs
\cite{%
  1510.06289v2,%
  PhysRevLett.113.097001%
},
where Fermi surfaces of each spin are present in each valley.
Our focus is the regime of maximal loss of spin degeneracy where the
effects are most striking
\cite{1604.02134v2}.
The two valleys in the energy landscape generically allow
two classes of superconducting phases:
intervalley pairing with zero center of mass momentum,
and intravalley pairing with finite Cooper pair center of mass.
Since center-of-symmetry is broken and spin degeneracy is lost,
classifications of superconducting states by parity,
i.e., singlet vs.\ triplet, is no longer possible.
In this paper, we study both extrinsic and intrinsic superconductivity
by projecting the interactions and pairing potential to
the topmost valence band.
We identify the possible phases, and analyze the nature
of the optoelectronic coupling and the response to magnetic fields.
Our main conclusions are as follows:

\introparanum{}
For both proximity to an $s$-wave superconductor,
and due to local attractive density-density interactions,
the leading instability is due to an intervalley paired state,
where the Cooper pair is an equal mixture
of a spin singlet and the $m = 0$ spin triplet.

\introparanum{}
While the valley selectivity of the optical transition is suppressed,
it remains finite.
Consequently, the two quasiparticles
generated by pair-breaking circularly polarized light
are correlated such that one is in the valence band of one valley
and the conduction band of the other.
The valley and bands are determined by the polarity of incident light.

\introparanum{}
The quasiparticles generated in (2)
both have the same charge and Berry curvature.
Thus an anomalous Hall effect is anticipated
as the two travel in the same direction transverse to an applied electric field.

\introparanum{}
An in-plane magnetic field tilts the spin,
modifying the internal structure of the Cooper pair,
however, no pair-breaking is induced in the absence of scalar impurities.
The suppression of the effective interaction leads
to a parametric reduction of the transition temperature.
In the presence of scalar impurities, pair-breaking is enabled,
but the associated critical magnetic field is large.
