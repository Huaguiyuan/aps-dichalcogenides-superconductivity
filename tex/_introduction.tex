\section{Introduction}

The interplay of spin orbit interaction and electron-electron interaction
is a fertile new area of research where a number of new phases of matter
and novel phenomena have been theoretically conjectured
and experimentally realized.
A particularly exciting development is the discovery of
single layer transition metal group-VI dichalcogenides (TMDCs),
\ce{MX2} ($\ce{M} = \ce{Mo}, \ce{W}$
and $\ce{X} = \ce{S}, \ce{Se}, \ce{Te}$) which
are direct band gap semiconductors that have all the ingredients necessary
to explore this interplay.
While they share the hexagonal crystal structure of graphene,
they differ in three important aspects:
(1) inversion symmetry is broken leading to a gap in the spectrum
as opposed to Dirac nodes;
(2) spin is coupled with momenta, resulting in
a large splitting of the valence bands;
and (3) the two bands near the chemical potential mostly have
the transition metal $d$-orbitals character.

A striking consequence of the spin-orbit coupling
and inversion symmetry breaking is the nontrivial Berry curvature
associated with the bands near the valleys.
The Berry curvature engenders an effective intrinsic angular momentum
associated with the Bloch wave functions.
Remarkably, even though the atomic orbitals involved all have $d$ character,
spin preserving optical transitions between valence
and conduction bands are now possible,
Furthermore, the valley-dependent sign of
the Berry curvature leads to selective photoexcitation,
where right-circular polarization couples to one valley,
and left-circular to the other.
As a consequence, a number of valleytronic and spintronic applications
are enabled, and have attracted a lot of attention over the last few years.

Our primary interest is in exploiting
the band structure and valley-contrasting probe afforded by
the nontrivial topology in order to study and manipulate
correlated phenomena in these systems.
In particular, we focus on hole-doped systems,
where an experimentally accessible window in energy
is characterized by two disconnected pieces of
spin non-degenerate Fermi surfaces.
One can preferentially excite electrons from one or the other Fermi surfaces.
Since the spins are locked to their valley index,
these excitations have specific $s_z$
(where the $z$-axis is perpendicular to the two dimensional crystal).
We focus on the possible superconducting states and their properties.

The spin-valley locking and its consequence for superconductivity,
dubbed Ising superconductivity, has been previously studied
for heavily doped $p$ and $n$ type TMDs,
where Fermi surfaces of each spin are present in each valley.
Here, we focus on the interesting new regime where only one Fermi surface
exists in each valley.
The two valleys in the energy landscape generically allow
two classes of superconducting phases:
intervalley pairing with zero center of mass momentum,
and intravalley pairing with finite center of mass for the cooper pairs.
In the latter case, time-reversal is preserved because there are as many pairs
with $+\vc{K}$ as $-\vc{K}$, where $± \vc{K}$ is the location of the valleys.
Since center-of-symmetry is broken and spin degeneracy is lost,
one expects classifications of superconducting states by parity
(i.e., singlet vs.\ triplet) is no longer possible.
In this paper, we study both extrinsic and intrinsic superconductivity
by projecting the interactions and pairing potential to
the topmost valence band.
We identify the possible phases, and analyze the nature
of opto-electronic coupling and the response to magnetic fields.

Our main conclusions are as follows.
(1) For both proximity to an $s$-wave superconductor,
and due to local attractive density-density interactions,
the leading instability is do to an intervalley paired state,
where the cooper pair is an equal mixture of singlet and $m = 0$ triplet.
(2) While the valley selectivity of the optical transition is suppressed,
it remains finite,
Thus, circularly polarized light results in pair breaking, where a cooper pair
splits into one quasiparticle in the conduction band of one valley,
while its partner is in the valence band of the other.
Consequently, we predict an anomalous Hall effect of quasiparticles,
but unlike the parent states, both excitations are particle-like;
(3) An in plane magnetic field tilts the spin, leading to a modification
of the internal structure of the cooper pair,
but no pair breaking is induced in the absence of scalar impurities.
The effective attractive interaction is suppressed, leading
to a parametric suppression of the transition temperature.
In the presence of scalar impurities, pair breaking is enabled,
but the associated critical magnetic field is large.
