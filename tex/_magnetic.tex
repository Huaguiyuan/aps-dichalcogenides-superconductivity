\section{Magnetic}

We briefly discuss the pair-breaking effects
of the in-plane magnetic field and for non-magnetic disorder.
% TODO Add footnote.
(The details are presented elsewhere[footnote].)
Due to the large spin-orbit interaction and spin splitting $λ$,
the superconducting state exhibits anomalous magnetic response.
Unlike the conventional superconductors,
where a uniform magnetic fields leads to pair breaking due to spin paramagnetism,
the Zeeman coupling in a clean system modifies the spin structure
of the cooper pair and parametrically suppresses $T_{c}$.
The phenomena is caused by modifying the effective coupling rather than pair breaking.
While the lack of pair breaking, in compliance to Anderson's theorem,
for nonmagnetic impurities is recovered
due to preservance of time reversal symmetry,
the transition is indeed suppressed by a combination
of magnetic field and scalar impurities.
The pair-breaking effect is characterized by the parameter
\begin{equation}
  δ_c
  = \frac{1}{τ_0} {\left( \frac{μ_B H_c^∥}{λ} \right)}^2,
\end{equation}
where $μ_B$ is the Bohr magneton and $τ_0^{-1}$ is
the collision rate resulting from the scalar disorder potential.
Note that for a clean system where $\tau_{0} \rightarrow \infty$,
we recover the result that there is no pair breaking.

Assuming a continuous phase transition induced by the magnetic field,
the pair-breaking equation at temperature $T$ is given by
\begin{equation}
  \log{\of{\frac{T_c^0}{T}}}
  = ψ \of{\frac{1}{2} + \frac{δ_c}{2 π k_B T}}
  - ψ \of{\frac{1}{2}},
\end{equation}
where $ψ \of{z}$ is the digamma function,
$T_c^0$ is the transition temperature in the clean system,
and $k_B$ is the Boltzmann constant.
This equation determines the upper critical field
$H_c^∥ \of{T}$ as a function of temperature.
Clearly, the upper critical field is greatly enhanced
by the large spin-orbit interaction.
At zero temperature, an explicit expression for the upper
critical field can be obtained as
\begin{equation}
  μ_B H_c^∥ \of{0}
  = λ \sqrt{2 π k_B T_c^0 τ_0 e^{ψ \of{1 / 2}}}.
\end{equation}
