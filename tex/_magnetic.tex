\section{In-plane magnetic field and scalar disorder}

In this section we discuss the effects of in-plane magnetic fields
and non-magnetic impurities on the superconducting state. We consider
the lightly hole-doped monolayer TMDs in the regime where the Fermi
level crosses the upper valence bands and is well separated from the
lower valence bands. In this regime, the system is a spin-valley locking
system with the spin-opposite Fermi pocket at each valley. Without loss of generality, we adopt a simplified model taking
into account the valence bands only. In a quasi two-dimensional (2D)
system, an in-plane magnetic field couples to quasiparticles through
spin paramagnetism with negligible orbital interactions. Applying
a uniform in-plane magnetic field in the $x$ direction $\mathbf{B}=(\begin{array}{ccc}
B, & 0, & 0\end{array})$, the system is by the Hamiltonian ($\hbar=k_{B}=c=1$):
\begin{equation}
\mathcal{H}_{\tau}(\mathbf{k})=-\frac{k^{2}}{2m}-\mu+\tau E_{soc}\hat{s}_{z}+\mu_{B}B\hat{s}_{x}\label{eq: valley_Hamiltonian}
\end{equation}
which is acting on the valley-spin basis $\phi_{\tau}(\mathbf{k})=\left[\begin{array}{cc}
c_{\tau\uparrow}(\mathbf{k}), & c_{\tau\downarrow}(\mathbf{k})\end{array}\right]^{T}$, where $\mu$ is the chemical potential, $\tau=\pm$ is the valley
index, $\hat{s}$'s are Pauli matrices operating in spin space, $\mu_{B}$
is the Bohr magneton, and the dispersion relations of the upper and
lower valence bands have been approximated by a quadratic form with
an effective mass $m=E_{g}/\left(2a^{2}t^{2}\right)$ ($E_{g}$ is the large energy gap between the conduction and valence bands $E_{g}\gg E_{soc}$, $a$ and $t$ are defined under Eq. (\ref{eq:H_tau_0})) and the momentum $\mathbf{k}=\left(\begin{array}{cc}
k_{x}, & k_{y}\end{array}\right)$ measured from the corresponding valley center with $k=|\mathbf{k}|$.
Note that in this section we use $\mathbf{k}$ to represent momentum
measured from the corresponding valley center and $\mathbf{p}$ to
represent momentum measured from the Brillouin zone (BZ) center. We
use the notation that $c_{\tau s}^{\dagger}(\mathbf{k})$ ($c_{\tau s}(\mathbf{k})$)
creates (annihilates) a quasiparticle with momentum $\mathbf{k}$
and spin $s$ in the valley $\tau$, and $c_{s}^{\dagger}(\mathbf{p})$
($c_{s}(\mathbf{p})$) creates (annihilates) a quasiparticle with
momentum $\mathbf{p}$ and spin $s$. Hamiltonian $\mathcal{H}_{\tau}(\mathbf{k})$
has the spectrum $E_{\tau,u/l}=-\frac{k^{2}}{2m}-\mu\pm\sqrt{E_{soc}^{2}+\left(\mu_{B}B\right)^{2}}$
with $u$ for the upper ($+$) and $l$ for the lower ($-$) band
at each valley, and the eigen states: $\varphi_{\tau}(\mathbf{k})=\left[\begin{array}{cc}
c_{\tau,u}(\mathbf{k}), & c_{\tau,l}(\mathbf{k})\end{array}\right]^{T}$. $c_{\tau,u}$ and $c_{\tau,l}$ correspond to the quasiparticles
in the band basis which is related to the spin basis through a field-
and valley-dependent unitary transformation $U_{\tau}(b)$: $\varphi_{\tau}(\mathbf{k})=U_{\tau}(b)\phi_{\tau}(\mathbf{k})$,
where $b\equiv\frac{\mu_{B}B}{E_{soc}}$ is the dimensionless magnetic
field. Applying a uniform in-plane magnetic field shifts both the
upper (lower) valence bands at the two valleys
by the same amount, so that the perfect nesting condition between
the Fermi pockets at the two valleys remains. On the other hand, the
quasiparticle spin acquires a finite in-plane component, i.e., deviating
from $\pm z$ direction. Explicitly we have $\langle\tau,u\left|\hat{s}_{z}\right|\tau,u\rangle=\frac{\tau}{2\sqrt{1+b^{2}}}$,
$\langle\tau,l\left|\hat{s}_{z}\right|\tau,l\rangle=\frac{-\tau}{2\sqrt{1+b^{2}}}$,
and $\langle\tau,u\left|\hat{s}_{x}\right|\tau,u\rangle=\frac{b}{2\sqrt{1+b^{2}}}$,
$\langle\tau,l\left|\hat{s}_{x}\right|\tau,l\rangle=\frac{-b}{2\sqrt{1+b^{2}}}$.
Therefore, the quasiparticle spin tilts towards the field direction
in the upper valence bands and tilts against the field direction in
the lower valence bands at both valleys. The change of quasiparticle
spin orientation induced by the in-plane field modifies the internal
structure of the Cooper pair and affects the pairing strength as shown
below.

To evaluate the effect of the magnetic field on the superconductivity we follow the procedure used in section  \ref{sec_sc}. A local attractive density-density
interaction with pairing strength $v_{0}$ can be written as: $\mathcal{H}_{int}=-v_{0}\int d^{2}\mathbf{r}\rho(\mathbf{r})\rho(\mathbf{r})$,
with the quasiparticle density $\rho(\mathbf{r})=\sum_{s}c_{s}^{\dagger}(\mathbf{r})c_{s}(\mathbf{r})$
where $c_{s}^{\dagger}(\mathbf{r})$ ($c_{s}(\mathbf{r})$) is the
Fourier transform of $c_{s}^{\dagger}(\mathbf{p})$ ($c_{s}(\mathbf{p})$).
Transforming to momentum space and projecting onto the upper valence
bands, the pairing Hamiltonian has the form
\begin{equation}
\mathcal{H}_{p}=-v'(b)\sum_{\mathbf{k},\mathbf{k}'}c_{+}^{\dagger}(\mathbf{k})c_{-}^{\dagger}(-\mathbf{k})c_{-}(-\mathbf{k}')c_{+}(\mathbf{k}')\label{eq: pairing_Hamiltonian}
\end{equation}
where we have ignored the upper-band subscript $u$ in the operators
$c_{\tau,u}^{\dagger}$ and $c_{\tau,u}$. $v'(b)$ is the effective
pairing strength in the presence of in-plane magnetic field $v'(b)\equiv\frac{1}{1+b^{2}}v_{0}$.
The Hamiltonian $\mathcal{H}_{p}$ describes an inter-valley pairing
with a pairing strength $v'$ suppressed by the in-plane field. At
zero field, $v'=v_{0}$, $c_{+}^{\dagger}=c_{+\uparrow}^{\dagger}$ and
$c_{-}^{\dagger}=c_{-\downarrow}^{\dagger}$, the pairing occurs between
opposite spins. At finite fields, $v'<v_{0}$, the quasiparticle at valley
$+$ ($-$), represented by $c_{+}^{\dagger}$ ($c_{-}^{\dagger}$),
has its up (down) spin tilted towards the field direction. As a result,
the inter-valley pairing contains equal-spin pairing component in
the presence of in-plane field.

We write the mean-field Hamiltonian, using the Nambu-valley basis
$\Psi_{\mathbf{k}}=\left[\begin{array}{cccc}
c_{+}(\mathbf{k}), & c_{-}(\mathbf{k}), & c_{-}^{\dagger}(-\mathbf{k}), & -c_{+}^{\dagger}(-\mathbf{k})\end{array}\right]^{T}$, as
\begin{equation}
\mathcal{H}_{MF}(\mathbf{k})=\xi_{\mathbf{k}}\hat{\eta}_{z}-\Delta\hat{\eta}_{x}\label{eq: Mean_Field_Hamiltonian}
\end{equation}
where $\hat{\eta}$'s are Pauli matrices acting on Nambu (particle-hole)
space, $\xi_{\mathbf{k}}=-\frac{k^{2}}{2m}-\mu+\sqrt{E_{soc}^{2}+\left(\mu_{B}B\right)^{2}}$,
and the mean field $\Delta=v'(b)\sum_{\mathbf{k}'}\left\langle c_{-}(-\mathbf{k}')c_{+}(\mathbf{k}')\right\rangle $
describes an inter-valley pairing field, which we choose to be real
for convenience.

In a conventional 2D superconductor with spin-degenerate Fermi surface,
the applying of an in-plane magnetic field causes an energy splitting
between opposite-spin bands. This energy mismatch between opposite
spins gives rise to a pair-breaking effect in the clean system characterized
by the pair-breaking equation for temperature $T\leq T_{c}^{0}$ \cite{Maki01061964}:
\begin{equation}
\ln\frac{T_{c}^{0}}{T}=\frac{1}{2}\left[\psi(\frac{1}{2}+\frac{i\mu_{B}B_{c}}{2\pi T})+\psi(\frac{1}{2}-\frac{i\mu_{B}B_{c}}{2\pi T})\right]-\psi(\frac{1}{2}),\label{eq: conventional_pair_breaking_eq}
\end{equation}
where $T_{c}^{0}$ is the transition temperature at zero field in
the clean system and $\psi(z)$ is the digamma function. This equation
determines the critical field $B_{c}$ that destroys the superconducting
state at temperature $T\leq T_{c}^{0}$ from spin paramagnetism. Furthermore,
the scattering from non-magnetic impurities does not alter this pair-breaking
equation (\ref{eq: conventional_pair_breaking_eq}) such that the
critical field remains the same regardless of the presence of scalar
disorders \cite{Maki01061964}.

Unlike the conventional 2D superconductors, the two single-spin Fermi
pockets at different valleys in our system remains to be perfectly
nested without energy mismatch caused by spin paramagnetism, except
that the spins at the two pockets are no longer opposite with equal-spin
components induced by the field. These two differences give rise to
new features in the spin-valley locking system on the effects of in-plane
magnetic fields. First, in the clean limit, the presence of in-plane
magnetic fields does not lead to a pair-breaking effect for the lack
of energy mismatch but mildly suppresses the transition temperature
through the weakening of the pairing strength. The suppressed transition
temperature $T_{c}^{'}$ is related to zero-field transition temperature
$T_{c}^{0}$ as $T_{c}^{'}=T_{c}^{0}\exp\left[-\frac{b^{2}}{v_{0}N_{F}}\right]$
in the mean-field theory, where $N_{F}$ is the density of states
at the Fermi level. Second, the superconducting state is no longer
immune to the scalar disorder, since non-magnetic disorder potential
can cause inter-valley scattering due to the field-induced parallel-spin
components on the two pockets. This interplay between the in-plane
magnetic field and the scalar disorder leads to a pair-breaking effect.

In the presence of dilute randomly-distributed non-magnetic impurities
the Hamiltonian for short-range impurity potential is given by
\begin{equation}
\mathcal{H}_{imp}=\sum_{j}\int d^{2}\mathbf{r}U_{0}\delta(\mathbf{r}-\mathbf{R}_{j})\rho(\mathbf{r})\label{eq: impurity_H_real}
\end{equation}
where $\mathbf{R}_{j}$ is the position of the $j$th impurity and
$U_{0}$ is the disorder strength. Transforming to momentum space
and projecting onto the upper valence bands, $\mathcal{H}_{imp}$
can be written, using the Nambu-valley basis $\Psi_{\mathbf{k}}$,
as
\begin{equation}
\mathcal{H}_{imp}=\sum_{\mathbf{k}_{1},\mathbf{k}_{2}}\sum_{j}e^{i(\mathbf{k}_{1}-\mathbf{k}_{2})\cdot\mathbf{R}_{j}}\Psi_{\mathbf{k}_{1}}^{\dagger}\hat{U}\Psi_{\mathbf{k}_{2}}\label{eq: impurity_H_momentum}
\end{equation}
with the disorder scattering vertex $\hat{U}$ taking the form
\begin{equation}
\hat{U}=U_{0}\hat{\eta}_{z}+U_{0}\frac{b}{\sqrt{1+b^{2}}}\hat{\tau}_{x}\label{eq: disorder_scattering_vertex}
\end{equation}
where $\hat{\tau}$'s are Pauli matrices operating in valley space.
The first term in $\hat{U}$ corresponds to intra valley scattering
and the second term corresponds to inter valley scattering. Note that
we have ignored the factors $e^{i(\pm2\mathbf{K})\cdot\mathbf{R}_{j}}$
in the inter valley terms because $e^{i(2\mathbf{K})\cdot\mathbf{R}_{j}}$
and $e^{i(-2\mathbf{K})\cdot\mathbf{R}_{j}}$ will appear in pair
and then cancel each other in the diagrammatical calculation of self
energy.

The self energy due to impurity scattering after averaging over randomly-distributed
impurity configurations, in the first-order Born approximation, is
obtained as \cite{AbrikosovGorkov1961,maki1969superconductivity}
\begin{equation}
\hat{\Sigma}(\mathbf{k},i\omega_{n})=n_{imp}\int\frac{d^{2}\mathbf{k}'}{(2\pi)^{2}}\hat{U}\hat{\mathcal{G}}_{0}(\mathbf{k}',i\omega_{n})\hat{U}\label{eq: self_energy}
\end{equation}
where $n_{imp}$ is the impurity concentration, and $\hat{\mathcal{G}}_{0}$
is the Green's function matrix of the clean system $\hat{\mathcal{G}}_{0}(\mathbf{k}',i\omega_{n})=\left[i\omega_{n}-\xi_{\mathbf{k}'}\hat{\eta}_{z}+\Delta\hat{\eta}_{x}\right]^{-1}$
with the Matsubara frequencies $\omega_{n}=(2n+1)\pi T$. After integrating
out $\xi$ in the self-energy, the disorder renormalized Green's function
matrix $\hat{\mathcal{G}}=\left[\hat{\mathcal{G}}_{0}^{-1}-\hat{\Sigma}\right]^{-1}$
can be parametrized as
\begin{equation}
\hat{\mathcal{G}}(\mathbf{k},i\omega_{n})=\left[i\tilde{\omega}_{n}-\xi_{\mathbf{k}}\hat{\eta}_{z}+\tilde{\Delta}\hat{\eta}_{x}+iF(\omega_{n})\hat{\eta}_{z}\hat{\tau}_{x}\right]^{-1}\label{eq: renormalized_G}
\end{equation}
where the quantities $\tilde{\omega}_{n}$, $\tilde{\Delta}$, and
$F(\omega_{n})$ have the definitions
\begin{equation}
\tilde{\omega}_{n}\equiv\omega_{n}+\left(\frac{1}{2\tau_{1}}+\frac{1}{2\tau_{2}}\right)\frac{\omega_{n}}{\sqrt{\omega_{n}^{2}+\Delta^{2}}},\label{eq: renormalized_omega}
\end{equation}
\begin{equation}
\tilde{\Delta}\equiv\Delta+\left(\frac{1}{2\tau_{1}}-\frac{1}{2\tau_{2}}\right)\frac{\Delta}{\sqrt{\omega_{n}^{2}+\Delta^{2}}},\label{eq: renormalized_Delta}
\end{equation}
\begin{equation}
F(\omega_{n})\equiv\frac{1}{2\tau_{1}}\frac{\omega_{n}}{\sqrt{\omega_{n}^{2}+\Delta^{2}}}\frac{2b}{\sqrt{1+b^{2}}}.\label{eq: function_F}
\end{equation}
Here $1/\tau_{1}$ and $1/\tau_{2}$ are the collision rates corresponding
to the disorder-induced intra- and inter-valley scattering, respectively,
with the expressions
\begin{equation}
\frac{1}{\tau_{1}}=2U_{0}^{2}n_{imp}\pi N_{F},\ \ \frac{1}{\tau_{2}}=\frac{1}{\tau_{1}}\frac{b^{2}}{1+b^{2}}.\label{eq: collision_rates}
\end{equation}


In the superconducting state, the self-consistency equation for the
order parameter is given by
\begin{equation}
\Delta=\frac{1}{4}v'T\sum_{n}\int\frac{d^{2}\mathbf{k}}{(2\pi)^{2}}\mathrm{Tr}\left[\hat{\eta}_{x}\hat{\mathcal{G}}(\mathbf{k},i\omega_{n})\right]\label{eq: self_consistency_eq}
\end{equation}
where $\mathrm{Tr}\left[\dots\right]$ is the trace of the argument.
Explicitly it has the form, from Eq. (\ref{eq: renormalized_G}),
\begin{equation}
\Delta=v'\pi N_{F}T\sum_{n}\frac{\tilde{\Delta}}{\sqrt{\tilde{\omega}_{n}^{2}+\tilde{\Delta}^{2}}}.\label{eq: explicit_self_consistency_eq}
\end{equation}
Linearizing the self-consistency equation (\ref{eq: explicit_self_consistency_eq})
near the critical field $B_{c}$, we obtain the pair-breaking equation
due to the interplay between the in-plane field and the scalar disorder,
\begin{equation}
\ln\frac{T_{c}^{'}(b_{c})}{T}=\psi\left(\frac{1}{2}+\frac{\delta_{c}}{2\pi T}\right)-\psi\left(\frac{1}{2}\right)\label{eq: pair_breaking_eq}
\end{equation}
where $T_{c}^{'}(b_{c})$ is the transition temperature in the clean
system in the presence of field $b_{c}$: $T_{c}^{'}(b_{c})=T_{c}^{0}\exp\left[-\frac{b_{c}^{2}}{v_{0}N_{F}}\right]$,
and the pair-breaking parameter $\delta_{c}$ arises from the valley-flip
scattering process,
\begin{equation}
\delta_{c}\equiv\frac{1}{\tau_{2}}\biggl|_{b_{c}}=\frac{1}{\tau_{1}}\frac{b_{c}^{2}}{1+b_{c}^{2}},\ \ b_{c}\equiv\frac{\mu_{B}B_{c}}{E_{soc}}.\label{eq: parameter_pair_breaking}
\end{equation}
Equation (\ref{eq: pair_breaking_eq}) determines the in-plane critical
field $B_{c}(T)$ at temperature $T\leq T_{c}^{'}$. For $b_{c}\equiv\frac{\mu_{B}B_{c}}{E_{soc}}\ll1$,
the pair-breaking parameter takes the simple form $\delta_{c}\approx\frac{1}{\tau_{1}}\left(\mu_{B}B_{c}/E_{soc}\right)^{2}$.

As $T\rightarrow0$, the pair-breaking equation (\ref{eq: pair_breaking_eq})
can be approximated by the asymptotic expansion of the digamma function,
which leads to $2\pi T_{c}^{'}\exp\left[\psi(\frac{1}{2})\right]=\frac{1}{\tau_{1}}\frac{b_{c}^{2}}{1+b_{c}^{2}}.$
At finite disorder concentration $\frac{1}{\tau_{1}}\neq0$, when
$b_{c}\ll1$ with $T_{c}^{'}\approx T_{c}^{0}$, the critical field at zero temperature is approximated as
\begin{equation}
\mu_{B}B_{c}\biggl|_{T\rightarrow0}\approx E_{soc}\left[2\pi e^{\psi(1/2)}k_{B}T_{c}^{0}\tau_{1}/\hbar\right]^{1/2}.\label{eq: zero_T_Bc}
\end{equation}
where we have put back the Boltzmann constant $k_{B}$ and
Planck constant $\hbar$. The large spin-orbit interaction $E_{soc}$
($\sim150-500$ meV) in monolayer TMDs indicates that the in-plane
critical field $B_{c}$ is significantly enhanced, well beyond the
Pauli limit.
