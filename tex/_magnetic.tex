\section{In-plane magnetic field and scalar disorder}

In this section we discuss the effects of in-plane magnetic fields
and non-magnetic impurities on the superconducting state.
We consider the lightly hole-doped monolayer TMDs in the regime
where the Fermi level crosses the upper valence bands
and is well separated from the lower valence bands.
In this regime, the system is a spin-valley locking
system with the spin-opposite Fermi pocket at each valley.
Without loss of generality, we adopt a simplified model taking
into account the valence bands only.
In a quasi two-dimensional (2D) system,
an in-plane magnetic field couples to quasiparticles through
spin paramagnetism with negligible orbital interactions.
Applying a uniform in-plane magnetic field in the $x$ direction
$\vc{B} = \left( B, 0, 0 \right)$,
the system is described by the Hamiltonian ($ℏ = k_B = c = 1$)
\begin{equation}
  \mathcal{H}_τ \ofK
  = - \frac{k^2}{2 m} - μ + τ E_{\text{soc}} \hat{s}_z + μ_B B \hat{s}_x,
\end{equation}
which is acting on the valley-spin basis
$ϕ_τ \ofK = {\left( c_{τ ↑} \ofK, c_{τ ↓} \ofK \right)}^T$,
where $μ$ is the chemical potential, $τ = ±$ is the valley
index, $\hat{s}_x, \hat{s}_y$ are Pauli matrices operating in spin space,
$μ_B$ is the Bohr magneton, and the dispersion relations of the upper and
lower valence bands have been approximated by a quadratic form with
an effective mass
$m = E_g / \left( 2 a^2 t^2 \right)$
($E_g$ is the large energy gap between the conduction and valence bands,
$E_g ≫ E_{\text{soc}}$,
$a$ and $t$ are defined under \cref{eq:hamiltonian}),
and the momentum $\vK = \left( k_x, k_y \right)$
is measured from the corresponding valley center with $k = \abs{\vK}$.
Note that in this section we use $\vK$ to represent momentum
measured from the corresponding valley center and $\vc{p}$ to
represent momentum measured from the Brillouin zone (BZ) center.
We use the notation that $c_{τ \s}^† \ofK$ ($c_{τ \s} \ofK$)
creates (annihilates) a quasiparticle with momentum $\vK$
and spin $\s$ in the valley $τ$, and $c_s^† \of{\vc{p}}$
($c_{\s} \of{\vc{p}}$) creates (annihilates) a quasiparticle with
momentum $\vc{p}$ and spin $\s$.

The Hamiltonian $\mathcal{H}_τ \ofK$ has the spectrum
\begin{equation}
  E_{τ, u / l} \of{k}
  = - \frac{k^2}{2 m} -μ ± \sqrt{E_{\text{soc}}^2 + {\left( μ_B B \right)}^2},
\end{equation}
with $u$ for the upper ($+$) and $l$ for the lower ($-$) band at each valley,
and the eigenstates
$φ_τ \ofK = {\left( c_{τ u} \ofK, c_{τ l} \ofK \right)}^T$,
where $c_{τ u}$ and $c_{τ l}$ correspond to the quasiparticles
in the band basis which is related to the spin basis through a field
and valley-dependent unitary transformation $U_τ \of{b}$:
$φ_τ \ofK = U_τ \of{b} ϕ_τ \ofK$,
where $b = μ_B B / {E_{\text{soc}}}$
is the dimensionless magnetic field.
Applying a uniform in-plane magnetic field shifts both the
upper (lower) valence bands at the two valleys
by the same amount, so that the perfect nesting condition between
the Fermi pockets at the two valleys remains.
On the other hand, the quasiparticle spin acquires a finite in-plane component,
i.e., deviating from $± z$ direction.
Explicitly we have
$\Braket{τ, u | \hat{s}_z | τ, u} = \left( τ / 2 \right) / \sqrt{1 + b^2}$,
$\Braket{τ, l | \hat{s}_z | τ, l} = - \left( τ / 2 \right) / \sqrt{1 + b^2}$,
and
$\Braket{τ, u | \hat{s}_x | τ, u} = \left( b / 2 \right) / \sqrt{1 + b^2}$,
$\Braket{τ, l | \hat{s}_x | τ, l} = - \left( b / 2 \right) / \sqrt{1 + b^2}$.
Therefore, the quasiparticle spin tilts towards the field direction
in the upper valence bands and tilts against the field direction in
the lower valence bands at both valleys.
The change of quasiparticle spin orientation induced by the in-plane field
modifies the internal structure of the Cooper pair
and affects the pairing strength as shown below.

To evaluate the effect of the magnetic field on the superconductivity
we follow the procedure used in \cref{s:superconductivity}.
A local attractive density-density
interaction with pairing strength $v_0$ can be written as
\begin{equation}
  \mathcal{H}_{\text{int}}
  = - v_0 ∫ ρ \of{\vc{r}} ρ \left( \vc{r}' \right)
    \: \dif^2 \vc{r} \: \dif^2 \vc{r}',
\end{equation}
with the quasiparticle density
$ρ \of{\vc{r}} = ∑_{\s} c_{\s}^† \of{\vc{r}} c_{\s} \of{\vc{r}}$
where $c_{\s}^† \of{\vc{r}}$ ($c_{\s} \of{\vc{r}}$) is the
Fourier transform of $c_{\s}^† \of{\vc{p}}$ ($c_{\s} \of{\vc{p}}$).
Transforming to momentum space and projecting onto the upper valence
bands, the pairing Hamiltonian has the form
\begin{equation}
  \mathcal{H}_p
  = - v' \of{b} ∑_{\vK, \vK'}
    c_+^† \ofK c_-^† \ofMK
    c_- \of{-\vK'} c_+ \of{\vK'},
\end{equation}
where we have ignored the upper-band subscript $u$ in the operators
$c_{τ, u}^†$ and $c_{τ, u}$.
The effective pairing strength in the presence of in-plane magnetic field is
$v' \of{b} = v_0 / \left( 1 + b^2 \right)$.
The Hamiltonian $\mathcal{H}_p$ describes an inter-valley pairing
with a pairing strength $v'$ suppressed by the in-plane field.
At zero field, $v' = v_0$,
$c_+^† = c_{+ ↑}^†$, and $c_-^† = c_{- ↓}^†$
so the pairing occurs between opposite spins.
At finite fields, $v' < v_0$, the quasiparticle at valley
$+$ ($-$), represented by $c_+^†$ ($c_-^†$),
has its up (down) spin tilted towards the field direction.
As a result, the inter-valley pairing contains equal-spin pairing component in
the presence of in-plane field.

We write the mean-field Hamiltonian, using the Nambu-valley basis
\begin{equation}
  Ψ_{\vK}
  = \left( c_+ \ofK, c_- \ofK, c_-^† \of{-\vK}, - c_+^† \of{-\vK} \right)^T
\end{equation}
as
\begin{equation}
  \mathcal{H}_{\text{MF}} \ofK
  = ξ_{\vK} \hat{η}_z - Δ \hat{η}_x
\end{equation}
where $\hat{η}_i$ are Pauli matrices acting on Nambu (particle-hole) space,
$ξ_{\vK}
= - \frac{k^2}{2m} - μ + \sqrt{E_{\text{soc}}^2 + \left( μ_B B \right)^2}$,
and the mean field
$Δ = v' \of{b} ∑_{\vK'} \left⟨ c_- \of{-\vK'} c_+ \of{\vK'} \right⟩$
describes an inter-valley pairing field, which we choose to be real
for convenience.

In a conventional 2D superconductor with spin-degenerate Fermi surface,
the applying of an in-plane magnetic field causes an energy splitting
between opposite-spin bands.
This energy mismatch between opposite spins gives rise
to a pair-breaking effect in the clean system characterized
by the pair-breaking equation for temperature $T ≤ T_c^0$
\cite{Maki01061964},
\begin{multline}
  \label{eq:conventional_pair_breaking_eq}
  \ln{\frac{T_c^0}{T}} +  ψ \of{\frac{1}{2}} \\
  = \frac{1}{2} \left[ ψ \of{\frac{1}{2} + \frac{i μ_B B_c}{2 π T}}
  + ψ \of{\frac{1}{2} - \frac{i μ_B B_c}{2 π T}} \right],
\end{multline}
where $T_c^0$ is the transition temperature at zero field in
the clean system and $ψ \of{z}$ is the digamma function.
This equation determines the critical field $B_c$
that destroys the superconducting state at temperature $T ≤ T_c^0$
from spin paramagnetism.
Furthermore, the scattering from non-magnetic impurities
does not alter this pair-breaking
\cref{eq:conventional_pair_breaking_eq} such that the
critical field remains the same regardless of the presence of scalar
disorders
\cite{Maki01061964}.

Unlike the conventional 2D superconductors, the two single-spin Fermi
pockets at different valleys in our system remains to be perfectly
nested without energy mismatch caused by spin paramagnetism, except
that the spins at the two pockets are no longer opposite with equal-spin
components induced by the field.
These two differences give rise to new features
in the spin-valley locking system on the effects of in-plane magnetic fields.
First, in the clean limit, the presence of in-plane
magnetic fields does not lead to a pair-breaking effect for the lack
of energy mismatch but mildly suppresses the transition temperature
through the weakening of the pairing strength.
The suppressed transition
temperature $T_c'$ is related to zero-field transition temperature $T_c^0$ as
$T_c' = T_c^0 \exp{\left( - b^2 / v_0 N_F \right)}$
in the mean-field theory,
where $N_F$ is the density of states at the Fermi level.
Second, the superconducting state is no longer
immune to the scalar disorder, since non-magnetic disorder potential
can cause inter-valley scattering due to the field-induced parallel-spin
components on the two pockets.
This interplay between the in-plane magnetic field
and the scalar disorder leads to a pair-breaking effect.

In the presence of dilute randomly-distributed non-magnetic impurities
the Hamiltonian for short-range impurity potential is given by
\begin{equation}
  \mathcal{H}_{\text{imp}}
  = ∑_j ∫ U_0 δ \of{\vc{r} - \vc{R}_j} ρ \of{\vc{r}}
    \: \dif^2 \vc{r},
\end{equation}
where $\vc{R}_j$ is the position of the $j$th impurity and
$U_0$ is the disorder strength.
Transforming to momentum space
and projecting onto the upper valence bands, $\mathcal{H}_{\text{imp}}$
can be written using the Nambu-valley basis $Ψ_{\vK}$ as
\begin{equation}
  \mathcal{H}_{\text{imp}}
  = ∑_{{\vK}_1, {\vK}_2} ∑_j
    e^{i \left( {\vK}_1 - {\vK}_2 \right) · \vc{R}_j}
    Ψ_{{\vK}_1}^† \hat{U} Ψ_{{\vK}_2},
\end{equation}
with the disorder scattering vertex $\hat{U}$ taking the form
\begin{equation}
  \hat{U}
  = U_0 \hat{η}_z + U_0 \frac{b}{\sqrt{1 + b^2}} \hat{τ}_x,
\end{equation}
where $\hat{τ}_i$ are Pauli matrices operating in valley space.
The first term in $\hat{U}$ corresponds to intra valley scattering
and the second term corresponds to inter valley scattering.
Note that we have ignored the factors
$e^{± 2 i \vK · \vc{R}_j}$
in the inter valley terms because
$e^{2 i \vK · \vc{R}_j}$ and $e^{-2 i \vK · \vc{R}_j}$
will appear in pair and then cancel each other
in the diagrammatical calculation of self energy.

The self energy due to impurity scattering
after averaging over randomly-distributed impurity configurations,
in the first-order Born approximation, is obtained as
\cite{AbrikosovGorkov1961,maki1969superconductivity}
\begin{equation}
  \hat{Σ} \of {\vK, i ω_n}
  = \frac{n_{\text{imp}}}{{\left( 2π \right)}^2}
    ∫ \hat{U} \hat{\mathcal{G}}_0 \of{\vK', i ω_n} \hat{U}
    \: \dif^2 \vK'
\end{equation}
where $n_{\text{imp}}$ is the impurity concentration,
and $\hat{\mathcal{G}}_0$
is the Green's function matrix of the clean system
$\hat{\mathcal{G}}_0 \of{\vK', i ω_n}
= {\left( i ω_n - ξ_{\vK'} \hat{η}_z + Δ \hat{η}_x \right )}^{-1}$
with the Matsubara frequencies
$ω_n = \left( 2 n + 1 \right) π T$.
After integrating out $ξ$ in the self-energy,
the disorder renormalized Green's function matrix
$\hat{\mathcal{G}} = {\left( \hat{\mathcal{G}}_0^{-1} - \hat{Σ} \right)}^{-1}$
can be parametrized as
\begin{equation}
  \label{eq:renormalized_g}
  \hat{\mathcal{G}} \of{\vK, i ω_n}
  = {\left[ i \tilde{ω}_n - ξ_{\vK} \hat{η}_z
    + \tilde{Δ} \hat{η}_x + i F \of{ω_n} \hat{η}_z \hat{τ}_x \right]}^{-1}
\end{equation}
where the quantities
$\tilde{ω}_n$, $\tilde{Δ}$, and $F \of{ω_n}$ have the definitions
\begin{equation}
  \tilde{ω}_n
  = ω_n + \left( \frac{1}{2 τ_1} + \frac{1}{2 τ_2} \right)
    \frac{ω_n}{\sqrt{ω_n^2 + Δ^2}},
\end{equation}
\begin{equation}
  \tilde{Δ}
  = Δ + \left( \frac{1}{2 τ_1} - \frac{1}{2 τ_2} \right)
    \frac{Δ}{\sqrt{ω_n^2 + Δ^2}},
\end{equation}
\begin{equation}
  F \of{ω_n}
  = \frac{1}{2 τ_1} \frac{ω_n}{\sqrt{ω_n^2 + Δ^2}} \frac{2b}{\sqrt{1 + b^2}}.
\end{equation}
Here $1 / τ_1$ and $1 / τ_2$ are the collision rates corresponding
to the disorder-induced intra- and inter-valley scattering, respectively,
with the expressions
\begin{equation}
  \frac{1}{τ_1} = 2 π U_0^2 N_F n_{\text{imp}}, \:
  \frac{1}{τ_2} = \frac{1}{τ_1} \frac{b^2}{1 + b^2}.
\end{equation}

In the superconducting state, the self-consistency equation for the
order parameter is given by
\begin{equation}
  Δ
  = \frac{1}{4} \frac{v' T}{{\left( 2 π \right)}^2}
    ∑_n ∫ \Tr{\left[ \hat{η}_x \hat{\mathcal{G}} \of{\vK, i ω_n} \right]}
    \: \dif^2 \vK.
\end{equation}
Explicitly it has the form, from \cref{eq:renormalized_g},
\begin{equation}
  \label{eq:explicit_self_consistency_eq}
  Δ = v' π N_F T ∑_n \frac{\tilde{Δ}}{\sqrt{\tilde{ω}_n^2 + \tilde{Δ}^2}}.
\end{equation}
Linearizing the self-consistency \cref{eq:explicit_self_consistency_eq}
near the critical field $B_c$, we obtain the pair-breaking equation
due to the interplay between the in-plane field and the scalar disorder,
\begin{equation}
  \label{eq:pair_breaking_eq}
  \ln{\frac{T_c' \of{b_c}}{T}}
  = ψ \of{\frac{1}{2} + \frac{δ_c}{2 π T}} - ψ \of{\frac{1}{2}},
\end{equation}
where $T_c' \of{b_c}$ is the transition temperature in the clean
system in the presence of field $b_c$:
$T_c' \of{b_c} = T_c^0 \exp{\left( - b_c^2 / v_0 N_F \right)}$,
and the pair-breaking parameter $δ_c$ arises from the valley-flip
scattering process,
\begin{equation}
  δ_c = \frac{1}{τ_2} \biggl{\rvert}_{b_c}
      = \frac{1}{τ_1} \frac{b_c^2}{1 + b_c^2}, \:
  b_c = \frac{μ_B B_c}{E_{\text{soc}}}.
\end{equation}
\Cref{eq:pair_breaking_eq} determines the in-plane critical
field $B_c \of{T}$ at temperature $T ≤ T_c'$.
For $b_c = \frac{μ_B B_c}{E_{\text{soc}}} ≪ 1$,
the pair-breaking parameter takes the simple form
$δ_c ≈ τ_1^{-1} {\left( μ_B B_c / E_{\text{soc}} \right)}^2$.

As $T → 0$, the pair-breaking \cref{eq:pair_breaking_eq}
can be approximated by the asymptotic expansion of the digamma function,
which leads to
$2 π T_c' \exp{\left[ ψ \of{1 / 2} \right]}
= \frac{1}{τ_1} \frac{b_c^2}{1 + b_c^2}$.
At finite disorder concentration $τ_1^{-1} ≠ 0$, when
$b_c ≪ 1$ with $T_c' ≈ T_c^0$,
the critical field at zero temperature is approximated as
\begin{equation}
  μ_B B_c \biggl{\rvert}_{T → 0}
  ≈ E_{\text{soc}} {\left[ 2π e^{ψ \of{1/2}} k_B T_c^0 τ_1 / ℏ \right]}^{1/2}.
\end{equation}
where we have put back the Boltzmann constant $k_{B}$ and
Planck constant $ℏ$.
The large spin-orbit interaction $E_{\text{soc}}$
($∼$ \SIrange[range-phrase=--, range-units=single]{150}{500}{\milli \electronvolt})
in monolayer TMDs indicates that the in-plane
critical field $B_c$ is significantly enhanced, well beyond the
Pauli limit.
