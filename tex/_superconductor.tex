\section{Superconductivity}

We consider two approaches to realizing a superconducting state.
First, we assume a proximity induced state obtained by
layering a TMD on an $s$-wave superconductor.
Second, we study an intrinsic correlated phase arising
from density-density interactions.

We use $d^ν_{τ {\s}} \ofK$ as the annihilation operator
for tight-binding $d$-orbital states,
and $c^n_{τ {\s}} \ofK$ for the eigenstates of the non-interacting Hamiltonian,
$λ_{\vK}$ for the energy dispersion for Bogoliubov quasiparticles
and $Δ_{\vK}$ for the superconducting gap function.

\subsection{Induced State}

A proximity $s$-wave superconductor will inject Cooper pairs
according to %
\footnote{%
  Note that all sums over $\vK$ are restricted to $\left| \vK \right|$
  less than some cutoff that restricts the momentum to a single valley.
}
\begin{equation}
  H^V
  = ∑_{\vK, ν, τ} \cc{B}_ν
    d^ν_{-τ ↓} \ofMK d^ν_{τ ↑} \ofK + \frac{ε}{2} + \hc
\end{equation}
The coupling constants $B_ν$ and the overall constant $ε$
depend on the material interface.
Using the abbreviated notation
$c_{\vK α} = c^-_{τ {\s}} \ofK$,
with $α =\ ↑↓$ for $τ = {\s} = ±$,
projecting onto the upper valence bands yields,
\begin{multline}
  \label{eq:induced}
  P_{τ = {\s}}^{n = -} \left( H^0 + H^V - μ N \right)
  = ∑_{\vK, α} ξ_{\vK} c_{\vK α}^† c_{\vK α} \\
  - \sumK \left( \cc{Δ}_{\vK} c_{-\vK ↓} c_{\vK ↑}
  + Δ_{\vK} c_{\vK ↑}^† c_{-\vK ↓}^† \right)
  + ε,
\end{multline}
where $ξ_{\vK} = E_{+ ↑}^- \of{\abs{\vK}} - μ$ and
the effective BCS gap function is
\begin{equation}
  Δ_{\vK}
  = \frac{1}{2} \left( B_+ + B_- \right)
  + \frac{1}{2} \left( B_+ - B_- \right)
    \cos{θ_{\vK}},
\end{equation}
with $θ_{\vK} = θ_{+↑}^- \of{\abs{\vK}}$.
This form is identical to the standard BCS Hamiltonian with
an effective spin index $α$.
However, the spin state of the Cooper pair is an equal superposition
of the singlet and the $m = 0$ component of spin triplet.
The corresponding quasiparticle eigenstates are
$γ_{\vK α}
= α \cos{β_{\vK}} c_{\vK α} + \sin{β_{\vK}} c_{-\vK, -α}^†$,
with energies
$λ_{\vK} = ± \sqrt{ξ_{\vK}^2 + Δ_{\vK}^2}$,
where $\cos{2 β_{\vK}} = ξ_{\vK} / λ_{\vK}$,
Note that if $B_+ = B_-$,
then $Δ_{\vK}$ is a constant and independent of $\vK$.
Even when $B_+$ and $B_-$ are different,
the constant term dominates.
Before exploring the nature of this state,
we analyze the case of intrinsic superconductivity,
and show that the same state is energetically preferred.

\subsection{Intrinsic Phase}

For a local attractive density-density interaction
(e.g.\ one mediated by phonons), the potential is
$V ⋍ \frac{1}{2} ∑_{\vR, \vR'} v_{\vR \vR'}
\normalorder{n_{\vR} n_{\vR'}}$,
with $v_{\vR \vR'} = v_0 δ_{\vR \vR'}$
and $n_{\vR}$ the total Wannier electron density at lattice vector $\vR$.
Projecting onto states near the chemical potential gives
\begin{multline}
  \label{eq:channels}
  P_{τ = {\s}}^{n = -} \left( H^V \right)
  = \sumKK v \of{\vK' - \vK} \\
  × \left(
    A_{\vK \vK'}^2 c_{\vK' ↑}^† c_{-\vK' ↑}^† c_{-\vK ↑} c_{\vK ↑}
  + A_{\vK' \vK}^2 c_{\vK' ↓}^† c_{-\vK' ↓}^† c_{-\vK ↓} c_{\vK ↓}
    \right. \\ + \left.
      2 \abs{A_{\vK \vK'}}^2
      c_{\vK' ↑}^† c_{-\vK' ↓}^† c_{-\vK ↓} c_{\vK ↑}
    \vphantom{2 \abs{V_{\vK \vK'}}^2} \right),
\end{multline}
where
\begin{equation}
  A_{\vK \vK'}
  = e^{i \left( ϕ_{\vK'} - ϕ_{\vK} \right)}
    \sin{\frac{θ_{\vK'}}{2}} \sin{\frac{θ_{\vK}}{2}}
  + \cos{\frac{θ_{\vK'}}{2}} \cos{\frac{θ_{\vK}}{2}}.
\end{equation}
The first two terms in \cref{eq:channels} lead to intravalley pairing,
and the third to intervalley pairing.
We analyze the possible states within mean field theory.
The BCS order parameter is
\begin{equation}
  χ
  = v_0 \sumK \cc{g}_{\vK} \ev{c_{-\vK α'} c_{\vK α}},
\end{equation}
where the form of $g_{\vK}$ depends on the particular pairing channel.
The resulting Hamiltonian has the same form as the BCS Hamiltonian in
\cref{eq:induced}
but with an effective $Δ_{\vK} = g_{\vK} · χ$.
The intravalley pairing has three symmetry channels,
with the couplings given by
$2 g_{\vK} = 1 +  \cos{θ_{\vK}}$,
$\sqrt{2} e^{- i ϕ_{\vK}} g_{\vK} = \sin{θ_{\vK}}$
and $2 e^{- 2 i ϕ_{\vK}} g_{\vK} = 1 - \cos{θ_{\vK}}$.
For these channels, since
$\ev{c_{-\vK α} c_{\vK α}} = - \ev{c_{\vK α} c_{-\vK α}}$,
relabeling $\vK → -\vK$ in the sum gives $χ = 0$ %
\footnote{%
  For odd parity interactions, where $v \ofMK = -v \ofK$, the
  intravalley pairing is not excluded by symmetry.
  Specifically, repeating the calculation with this assumption,
  the intervalley terms fully cancel, and one obtains \cref{eq:channels}
  without the intervalley term on the third line.
}.
The intervalley pairing also has three symmetry channels:
$g_{\vK} = \sqrt{2}$,
$g_{\vK} = \sqrt{2} \cos{θ_{\vK}}$,
and $g_{\vK} = \sqrt{2} \sin{θ_{\vK}} \vc{\hat{k}}$.
Of the three,
the constant valued channel is dominant %
\footnote{%
  For example, using the values for \ce{WSe2},
  $\sin^2 {θ_{\vK}} = 0.44$ and $\cos^2 {θ_{\vK}} = 0.56$
  at the chemical potential.
}.
This is to be expected, as the local density-density interaction
leads to the largest pairing for electrons of opposite spins.
Since the intravalley processes have the same spin,
they are disfavored as compared to the intervalley pairing.

The key features of the intrinsic superconducting state
are identical to the proximally induced case when density-density
interactions dominate.
We restrict further analysis to that case,
and turn to the question of pair-breaking phenomena
induced either by optical or magnetic fields.
