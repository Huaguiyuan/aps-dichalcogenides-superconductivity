\section{Superconductivity}

We consider two approaches to realizing a superconducting state.
First, we assume a proximity induced state obtained by
layering a TMD on a suitable $s$-wave superconductor.
Second, we consider an intrinsic correlated phase arising
from density-density interactions.

In this section, we introduce the notation for the
Fock space creation and annihilation operators,
with $a^ν_{τ {\s}} \ofK$ corresponding to the tight-binding orbital states,
and $c^n_{τ {\s}} \ofK$ to the resulting eigenstates of the full Hamiltonian.

\subsection{Induced State}

A proximity $s$-wave superconductor will inject Cooper pairs
according to %
\footnote{%
  Note that all sums over $\vK$ are restricted to $\left| \vK \right|$
  less than some cutoff which restricts $\vc{K}$ to a single valley.
}
\begin{equation}
  H^V
  = ∑_{\vK, ν, τ} \cc{Δ}_ν
    a^ν_{-τ ↓} \ofMK a^ν_{τ ↑} \ofK + \frac{ε}{2} + \hc
\end{equation}
The coupling constants $Δ_ν$ and the overall constant $ε$
will depend on the material interface.
Using the abbreviated notation
$c_{\vK α} = c^-_{τ {\s}} \ofK$,
with $α = ↑,↓$ for $τ = {\s} = ±$,
we project this into the upper-valance bands and obtain the Hamiltonian,
\begin{multline}
  \label{eq:induced}
  P_{τ = {\s}}^{n = -} \left( H^0 + H^V - μ N \right)
  = ∑_{\vK, α} ξ_{\vK} c_{\vK α}^† c_{\vK α} \\
  - \sumK \left( \cc{Δ}_{\vK} c_{-\vK ↓} c_{\vK ↑}
  + Δ_{\vK} c_{\vK ↑}^† c_{-\vK ↓}^† \right)
  + ε,
\end{multline}
where $ξ_{\vK} = E_{+ ↑}^- \of{\abs{\vK}} - μ$ and
\begin{equation}
  Δ_{\vK}
  = \frac{1}{2} \left( Δ_+ + Δ_- \right)
  + \frac{1}{2} \left( Δ_+ - Δ_- \right)
    \cos{θ_{\vK}},
\end{equation}
with $θ_{\vK} = θ_{+↑}^- \of{\abs{\vK}}$.
This form is identical to the standard BCS Hamiltonian with
an effective spin index $α$.
However, the spin state of the cooper pair is an equal superposition
of the singlet and the $m = 0$ component of spin triplet.
The corresponding energies,
$λ_{\vK} = ± \sqrt{ξ_{\vK}^2 + Δ_{\vK}^2}$,
and quasiparticle eigenstates,
$b_{\vK α}
= α \cos{β_{\vK}} c_{\vK α} + \sin{β_{\vK}} c_{-\vK, -α}^†$,
with $\cos{2 β_{\vK}} = ξ_{\vK} / λ_{\vK}$,
follow immediately.
Note that $Δ_{\vK}$ is a constant and independent of $\vK$
for $Δ_+ = Δ_-$.
Even when $Δ_+$ and $Δ_-$ are different,
the constant term is dominant.
Before exploring the nature of this state,
we analyze the case of intrinsic superconductivity,
and show that the same state is energetically preferred.

\subsection{Intrinsic Phase}

For a local attractive density-density interaction
(e.g.\ one mediated by phonons), the potential is
$V ⋍ \frac{1}{2} ∑_{\vR, \vR'} v_{\vR \vR'}
\normalorder{n_{\vR} n_{\vR'}}$,
with $v_{\vR \vR'} = v_0 δ_{\vR \vR'}$
and $n_{\vR}$ the total Wannier electron density at lattice vector $\vR$.
Projecting to the states near the chemical potential gives
\begin{multline}
  \label{eq:channels}
  P_{τ = {\s}}^{n = -} \left( H^V \right)
  = \sumKK v \of{\vK' - \vK} \\
  × \left(
    A_{\vK \vK'}^2 c_{\vK' ↑}^† c_{-\vK' ↑}^† c_{-\vK ↑} c_{\vK ↑}
  + A_{\vK' \vK}^2 c_{\vK' ↓}^† c_{-\vK' ↓}^† c_{-\vK ↓} c_{\vK ↓}
    \right. \\ + \left.
      2 \abs{A_{\vK \vK'}}^2
      c_{\vK' ↑}^† c_{-\vK' ↓}^† c_{-\vK ↓} c_{\vK ↑}
    \vphantom{2 \abs{V_{\vK \vK'}}^2} \right),
\end{multline}
where
\begin{equation}
  A_{\vK \vK'}
  = e^{i \left( ϕ_{\vK'} - ϕ_{\vK} \right)}
    \sin{\frac{θ_{\vK'}}{2}} \sin{\frac{θ_{\vK}}{2}}
  + \cos{\frac{θ_{\vK'}}{2}} \cos{\frac{θ_{\vK}}{2}}.
\end{equation}
The first two terms in \cref{eq:channels} lead to intravalley pairing,
and the third to intervalley pairing.
We analyze the possible states within mean field.
The order parameter is
\begin{equation}
  Δ_0
  = v_0 \sumK \cc{g}_{\vK} \ev{c_{-\vK α'} c_{\vK α}},
\end{equation}
where the form of $g_{\vK}$ depends on the particular pairing channel.
The resulting Hamiltonian has the same form as the BCS Hamiltonian in
\cref{eq:induced}
with an effective $Δ_{\vK} = g_{\vK} · Δ_0$.
The intravalley pairing has three symmetry channels,
with the couplings given by
$2 g_{\vK} = 1 +  \cos{θ_{\vK}}$,
$\sqrt{2} e^{- i ϕ_{\vK}} g_{\vK} = \sin{θ_{\vK}}$
and $2 e^{- 2 i ϕ_{\vK}} g_{\vK} = 1 - \cos{θ_{\vK}}$.
For these channels, since
$\ev{c_{-\vK α} c_{\vK α}} = - \ev{c_{\vK α} c_{-\vK α}}$,
relabeling $\vK → -\vK$ in the sum gives $Δ_0 = 0$ %
\footnote{%
  For complex interactions, where
  $v \of{\vK} ≠ v \of{-\vK}$,
  the intravalley order parameters will in general be nonzero.
}.
The intervalley pairing also has three symmetry channels:
$g_{\vK} = \sqrt{2}$,
$g_{\vK} = \sqrt{2} \cos{θ_{\vK}}$,
and $g_{\vK} = \sqrt{2} \sin{θ_{\vK}}$.
Of the three,
the constant valued channel is dominant %
\footnote{%
  For example, using the values for \ce{WSe2},
  $\sin^2 {θ_{\vK}} = 0.44$ and $\cos^2 {θ_{\vK}} = 0.56$
  at the chemical potential.
}.

This is to be expected as the local density-density interaction
leads to the largest pairing for electrons of opposite spins.
Since the intravalley processes have the same spin,
they are disfavored as compared to the intervalley pairing.
This result further suggests that intravalley pairing requires
odd parity couplings which may arise from spin exchange interactions.
Thus, the key features of the intrinsic superconducting state
are identical to the proximally induced case, and we restrict further
analysis to that case.
We now turn to the question of pair breaking phenomena
induced either by optical or magnetic fields.
