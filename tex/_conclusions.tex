\section{Conclusions}

In this letter, we report on the nature of the superconducting state
of hole-doped TMDCs.
Remarkably, the correlated state inherits
the valley contrasting phenomena of the parent state.
While the magnitude is smaller, pair breaking produces quasiparticles
that have the same Berry curvature, and hence the same anomalous velocity.
Thus, one predicts an anomalous hall response similar to
the one observed in in \ce{MoSe2}.
Further consequences of the phenomena will be explored elsewhere.

Spin-valley locking leads to large critical magnetic fields.
A similar phenomena was recently reported in heavily hole-doped
(beyond the spin-split gap) \ce{NbSe2}.
In the new regime, where only one band per valley intersects
the chemical potential, no pair breaking occurs
for in-plane fields unless disorder is present.

While systematic synthesis and characterization of hole-doped systems
is still in its early stages, the fact that other two-dimensional compounds
and their bulk counterparts are known to be superconducting
provides impetus to explore the novel phenomena described here.

The software developed and used for this work
is available freely online
\footnote{%
  Related software and source code at \\
  \url{https://evansosenko.com/dichalcogenides/}
}.
We acknowledge the support of Army Research Office through the grant
ARO W911NF1510079.
